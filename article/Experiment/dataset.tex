\subsection{Datasets}
For our experiments, we have used 3 different datasets.
These are \textit{yelp-2020}, \textit{amazon-book} and \textit{amazon-cell-sport}.
The data split is 20\% testing and 80\% training where 10\% of the training data is used as validation data.
A comparison of the datasets can be seen on \autoref{tab:dataset-comparison}.
\\
\textbf{The \textit{amazon-book} dataset} is the largest dataset we used.
It is taken from the \cite{lightgcn} paper and only contains ID's for users and items.
As in \cite{lightgcn} we have used the 10-core setting \textit{i.e.} only removing users and items with less than ten interactions.
\\
\textbf{The \textit{amazon-cell-sport} dataset} is taken from \cite{BiTGCF}.
This dataset contains two domains \textit{cell} short for Cell Phones and \textit{sport} short for Sports and Outdoors.
It contains two domains because \cite{BiTGCF} builds an extension of \cite{lightgcn} which utilizes a cross-domain recommendation technique.
To build a cross-domain dataset first all the overlapping users were found.
Then all the users with five or more interactions were retained.
Items from each domain have both been split into training and testing before being merged.
\\
\textbf{The \textit{yelp-2020} dataset} was made as a compromise to test both \cite{lightgcn} and \cite{PUP}.
PUP needs both price and category in addition to user and item IDs so a dataset was needed which could be used to validate both methods.
Both methods used a different version of Yelp's datasets which were incompatible so utilizing Yelp's newest dataset (at the time) was decided.
Inspired by PUP the dataset only retains businesses belonging to the restaurant category where each restaurant have one or more sub-categories such as Mexican, Chinese, etc.
To make the data simpler only one sub-category is retained for each item, this is what is considered its category in the rest of the text.
In this dataset also we have also utilized the 10-core setting.


\begin{table*}[]
    \centering
    \begin{tabular}{|l|l|l|l|l|l|}
    \hline
                                & Users     & Items     & Items/users ratio   & Interactions      & Sparsity  \\ \hline
        Amazon-cell             & 4,998     & 14,618    & 2.924               & 47,444            & 99.9351\% \\ \hline
        Amazon-sport            & 4,998     & 22,101    & 4.421               & 55,556            & 99.9497\% \\ \hline
        Amazon-cell-sport       & 4,998     & 36,719    & 7.347               & 103,000           & 99.9438\% \\ \hline
        Amazon-book             & 52,643    & 91,599    & 1.74                & 2,984,108         & 99.9381\% \\ \hline
        Yelp2020                & 24,384    & 20,091    & 0.824               & 594,196           & 99.8787\% \\ \hline
    \end{tabular}
    \caption{Difference between the datasets}
    \label{tab:dataset-comparison}
\end{table*}