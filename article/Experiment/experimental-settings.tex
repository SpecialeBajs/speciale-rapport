\begin{table*}[h!]
    \scriptsize
    \centering
    \begin{tabular}{|l|l|l|l|l|l|}
        \hline
                                & Users  & Items   & Items/users ratio & Interactions & Sparsity  \\ \hline
        Amazon-Book             & 52,643 & 91,599  & 1.74              & 2,984,108    & 99.9381\% \\ \hline
        Yelp2020                & 24,384 & 20,091  & 0.824             & 594,196      & 99.8787\% \\ \hline
        Amazon-Cell-Sport       & 4,998  & 36,719  & 7.347             & 103,000      & 99.9438\% \\ \hline
        Amazon-Cell-Electronic  & 3,325  & 57,925  & 17.42             & 172,611      & 99.9103\% \\ \hline
        Amazon-Cloth-Electronic & 15,761 & 105,174 & 6.673             & 363,474      & 99,9780\% \\ \hline
        Amazon-Cloth-Sport      & 9,928  & 73,613  & 7.414             & 200,297      & 99,9726\% \\ \hline
    \end{tabular}
    \caption{Comparisons on the datasets}
    \label{tab:dataset-comparison}
\end{table*}


\subsection{Experimental Settings}

In this subsection is the datasets and the settings that has been used for the experiments outlined.

\subsubsection{Datasets}

For our experiments, we have used 6 different datasets.
These are \textit{Yelp2020}, \textit{Amazon-Book}, \textit{Amazon-Cell-Sport}, \textit{Amazon-Cell-Electronic}, \textit{Amazon-Cloth-Electronic} and \textit{Amazon-Cloth-Sport}.
The data split is 20\% testing and 80\% training where 10\% of the training data is used as validation data.
A comparison of the datasets can be seen on \autoref{tab:dataset-comparison}.
\\
Each dataset has some applied k-core settings to it.
K-core means that all users and items with less than \textit{k} connections are removed.
This is because users and items with too few connections wont have enough data to compare to other users and items and therefore making collaborative filtering ineffective.
\\
\textbf{\textit{Amazon-Book}} is the largest dataset we have used.
It is taken from the LightGCN paper and only contains ID's for users and items \cite{lightgcn}.
LightGCN have used the 15-core setting for users and 5-core settings for items \textit{i.e.} removing users with less than 15 interactions and items with less than 5 interactions.
\\
\textbf{The \textit{Yelp2020} dataset} is a dataset we used in our pre-specialization paper to mimic a dataset used in LightGCN that we could not get access to.
This dataset uses 10-core settings for users and 5-core settings for items.
It is a subset of the full yelp2020 dataset where all entries are restaurants with one category and price attribute.
\\
\textbf{\textit{Amazon-Cell-Sport}, \textit{Amazon-Cell-Electronic}, \textit{Amazon-Cloth-Electronic} and \textit{Amazon-Cloth-Sport}} are taken from BiTGCF \cite{BiTGCF}.
These datasets contain two domains because BiTGCF builds an extension of LightGCN which utilizes a cross-domain transfer learning recommendation model \cite{BiTGCF}.
To build a cross-domain dataset first all the overlapping users were found.
Then all the users with five or more interactions were retained.
Items from each domain have both been split into training and testing before being merged.

\textit{Amazon-Cell-Sport}, \textit{Amazon-Cloth-Electronic} and \textit{Amazon-Cloth-Sport} use 5-core settings for users and 1-core settings for items.
There is an exception for \textit{Amazon-Cell-Electronic} which uses 15-core settings for users.


\paragraph{Interactions in the datasets}
As we expect there is a relation between the number of connections for each node and the performance of a convolution layer.
On \autoref{tab:dataset-item-and-user-splits} and \autoref{tab:node-degrees-cell-sport-electronic} each degree range of users and items can be seen as well as what amount of users and items each dataset have of each degree range.
Users in \textit{Amazon-Book} have an average node degree of 45.22 and items have an average node degree of 25.99 connections, which makes it the dataset with the highest average node degree compared to the other datasets.
For Yelp2020 the average node degree is 19.04 for users and 23.31 for items.
The other four amazon datasets have in common that the average node degree for items is small.
Users in \textit{Amazon-Cell-Sport} have an average node degree of 17.07 and items have an average node degree of 2.58.
\textit{Amazon-Cell-Electronic} has an average node degree of 42.22 for users and an average node degree 2.73 for items.
\textit{Amazon-Cloth-Electronic} has an average node degree of 19.04 for users and an average node degree of 3.12 for items.
\textit{Amazon-Cloth-Sport} has an average node degree of 16.72 for users and an average node degree of 2.49 for items.
\\
\\
These core-settings was chosen based on what the datasets in Bit-GCF and LightGCN used. 
This way we felt it was more fair to compare our method to the dataset that the other state of the art methods utilized.


\begin{table*}[]
    \scriptsize
    \begin{tabular}{|l|l|l|l|l|l|l|l|l|l|l|l|l|}
        \hline
                & \multicolumn{4}{c|}{Yelp2020} & \multicolumn{4}{c|}{Amazon-Cell-Sport} & \multicolumn{4}{c|}{Amazon-Book}                                                                                                                                                     \\ \hline
        Degree   & \multicolumn{2}{c|}{Users}    & \multicolumn{2}{c|}{Items}             & \multicolumn{2}{c|}{Users}       & \multicolumn{2}{c|}{Items} & \multicolumn{2}{c|}{Users} & \multicolumn{2}{c|}{Items}                                                              \\ \hline
        1-5     & 0                             & 0 \%                                   & 5185                             & 26.02 \%                   & 0                          & 0 \%                       & 30152 & 91.31 \%     & 0     & 0 \%     & 4393  & 4.79  \% \\ \hline
        6-10    & 7298                          & 29.92 \%                               & 4652                             & 23.34 \%                   & 1190                       & 23.80 \%                   & 1677  & 5.07 \%      & 0     & 0 \%     & 20562 & 22.44 \% \\ \hline
        11-15   & 7175                          & 29.42 \%                               & 2505                             & 12.57 \%                   & 1890                       & 37.81 \%                   & 565   & 1.71 \%      & 0     & 0 \%     & 21561 & 23.53 \% \\ \hline
        16-20   & 3862                          & 15.83 \%                               & 1622                             & 8.14 \%                    & 936                        & 18.72 \%                   & 270   & 0.817 \%     & 17098 & 32.47 \% & 12092 & 13.20 \% \\ \hline
        21-25   & 1742                          & 7.14 \%                                & 1135                             & 5.69 \%                    & 396                        & 7.92 \%                    & 137   & 0.41 \%      & 8234  & 15.64 \% & 7638  & 8.33 \%  \\ \hline
        26-30   & 1160                          & 4.75 \%                                & 814                              & 4.08 \%                    & 211                        & 4.22 \%                    & 78    & 0.23 \%      & 5257  & 9.98 \%  & 5092  & 5.55 \%  \\ \hline
        31-35   & 749                           & 3.07 \%                                & 624                              & 3.13 \%                    & 115                        & 2.30 \%                    & 44    & 0.133 \%     & 3722  & 7.07 \%  & 3791  & 4.13 \%  \\ \hline
        36-40   & 664                           & 2.72 \%                                & 487                              & 2.44 \%                    & 78                         & 1.56 \%                    & 27    & 0.081 \%     & 3154  & 5.99 \%  & 2891  & 3.15 \%  \\ \hline
        41-45   & 380                           & 1.55 \%                                & 398                              & 1.99 \%                    & 46                         & 0.92 \%                    & 19    & 0.057 \%     & 2029  & 3.85 \%  & 2301  & 2.51 \%  \\ \hline
        46-50   & 243                           & 0.99 \%                                & 323                              & 1.62 \%                    & 31                         & 0.62 \%                    & 14    & 0.042 \%     & 1591  & 3.02 \%  & 1775  & 1.93 \%  \\ \hline
        51-60   & 409                           & 1.67 \%                                & 475                              & 2.38 \%                    & 36                         & 0.72 \%                    & 13    & 0.039 \%     & 2581  & 4.90 \%  & 2407  & 2.62 \%  \\ \hline
        61-70   & 200                           & 0.82 \%                                & 316                              & 1.58 \%                    & 19                         & 0.38 \%                    & 5     & 0.015 \%     & 1664  & 3.16 \%  & 1662  & 1.81 \%  \\ \hline
        71-80   & 126                           & 0.51 \%                                & 271                              & 1.36 \%                    & 13                         & 0.26 \%                    & 6     & 0.018 \%     & 1379  & 2.61 \%  & 1183  & 1.29 \%  \\ \hline
        81-90   & 102                           & 0.41 \%                                & 199                              & 0.99 \%                    & 5                          & 0.10 \%                    & 4     & 0.012     \% & 954   & 1.81 \%  & 883   & 0.96 \%  \\ \hline
        91-100  & 71                            & 0.29 \%                                & 145                              & 0.72 \%                    & 6                          & 0.12 \%                    & 0     & 0 \%         & 745   & 1.41 \%  & 619   & 0.67 \%  \\ \hline
        101-150 & 127                           & 0.52 \%                                & 434                              & 2.17 \%                    & 23                         & 0.46 \%                    & 6     & 0.018 \%     & 2120  & 4.02 \%  & 1528  & 1.66 \%  \\ \hline
        151-200 & 52                            & 0.21 \%                                & 179                              & 0.89 \%                    & 2                          & 0.04 \%                    & 1     & 0.003     \% & 892   & 1.69 \%  & 553   & 0.60 \%  \\ \hline
        201-250 & 17                            & 0.06 \%                                & 64                               & 0.32 \%                    & 0                          & 0    \%                    & 0     & 0     \%     & 469   & 0.89 \%  & 251   & 0.27 \%  \\ \hline
        251-300 & 1                             & 0.004 \%                               & 36                               & 0.18 \%                    & 0                          & 0    \%                    & 0     & 0     \%     & 254   & 0.48 \%  & 146   & 0.15 \%  \\ \hline
        300+    & 7                             & 0.028 \%                               & 61                               & 0.30 \%                    & 1                          & 0.02 \%                    & 0     & 0  \%        & 500   & 0.94 \%  & 271   & 0.29 \%  \\ \hline
        Avg con & \multicolumn{2}{l|}{19.04}    & \multicolumn{2}{l|}{23.31}             & \multicolumn{2}{l|}{17.07}       & \multicolumn{2}{l|}{2.58}  & \multicolumn{2}{l|}{45.22} & \multicolumn{2}{l|}{25.99}                                                              \\ \hline
    \end{tabular}
    \caption{Amount of nodes within a certain node degree for Amazon-Cell-Sport, Amazon-Book and Yelp2020. The Avg connection shows how many connections each user or item have on average}
    \label{tab:dataset-item-and-user-splits}
\end{table*}

\begin{table*}[]
    \scriptsize
    \centering
    \begin{tabular}{|l|l|l|l|l|l|l|l|l|l|l|l|l|}
        \hline
                & \multicolumn{4}{c|}{Amazon-Cell-Electronic} & \multicolumn{4}{c|}{Amazon-Cloth-Electronic} & \multicolumn{4}{c|}{Amazon-Cloth-Sport}                                                                                                                                        \\ \hline
        Degree   & \multicolumn{2}{c|}{Users}                  & \multicolumn{2}{c|}{Items}                   & \multicolumn{2}{c|}{Users}              & \multicolumn{2}{c|}{Items} & \multicolumn{2}{c|}{Users} & \multicolumn{2}{c|}{Items}                                                 \\ \hline
        1-5     & \multicolumn{1}{r|}{0}                      & \multicolumn{1}{r|}{0.0 \%}                     & \multicolumn{1}{r|}{46183}              & \multicolumn{1}{r|}{89.94\%} & \multicolumn{1}{r|}{0}     & \multicolumn{1}{r|}{0.0 \%}   & 85169 & 88.81 \% & 0    & 0.0 \%   & 61217 & 91.85 \%\\ \hline
        6-10    & \multicolumn{1}{r|}{0}                      & \multicolumn{1}{r|}{0.0 \%}                     & \multicolumn{1}{r|}{3072}               & \multicolumn{1}{r|}{5.982\%} & \multicolumn{1}{r|}{3345}  & \multicolumn{1}{r|}{21.22 \%} & 6505  & 6.783 \% & 2331 & 23.479 \%& 3603  & 5.406 \%\\ \hline
        11-15   & \multicolumn{1}{r|}{0}                      & \multicolumn{1}{r|}{0.0 \%}                     & \multicolumn{1}{r|}{990}                & \multicolumn{1}{r|}{1.928\%} & \multicolumn{1}{r|}{5625}  & \multicolumn{1}{r|}{35.68 \%} & 1944  & 2.027 \% & 3675 & 37.016 \%& 979   & 1.468 \%\\ \hline
        16-20   & 538                                         & 16.180 \%                                      & 477                                     & 0.928 \%                     & 2957                       & 18.76 \%                     & 862   & 0.898 \% & 1853 & 18.664 \%& 364   & 0.546 \%\\ \hline
        21-25   & 685                                         & 20.601 \%                                      & 204                                     & 0.397 \%                     & 1358                       & 8.616 \%                     & 429   & 0.447 \% & 916  & 9.226 \% & 168   & 0.252 \%\\ \hline
        26-30   & 477                                         & 14.345 \%                                      & 128                                     & 0.249 \%                     & 774                        & 4.910 \%                     & 243   & 0.253 \%& 443  & 4.462 \% & 110   & 0.165 \%\\ \hline
        31-35   & 334                                         & 10.045 \%                                      & 63                                      & 0.122 \%                     & 458                        & 2.905 \%                     & 170   & 0.177 \%& 244  & 2.457 \% & 56    & 0.084 \%\\ \hline
        36-40   & 250                                         & 7.518 \%                                       & 61                                      & 0.118 \%                     & 295                        & 1.871 \%                     & 131   & 0.136 \%& 149  & 1.500 \% & 53    & 0.079 \%\\ \hline
        41-45   & 177                                         & 5.323 \%                                       & 42                                      & 0.081 \%                     & 208                        & 1.319 \%                     & 97    & 0.101 \%& 95   & 0.956 \% & 32    & 0.048 \%\\ \hline
        46-50   & 146                                         & 4.390 \%                                       & 29                                      & 0.056 \%                     & 157                        & 0.996 \%                     & 63    & 0.065 \%& 64   & 0.644 \% & 17    & 0.025 \%\\ \hline
        51-60   & 194                                         & 5.834 \%                                       & 40                                      & 0.077 \%                     & 186                        & 1.180 \%                     & 67    & 0.069 \%& 62   & 0.624 \% & 23    & 0.034 \%\\ \hline
        61-70   & 144                                         & 4.330 \%                                       & 16                                      & 0.031 \%                     & 107                        & 0.678 \%                     & 51    & 0.053 \%& 37   & 0.372 \% & 6     & 0.009 \%\\ \hline
        71-80   & 98                                          & 2.947 \%                                       & 12                                      & 0.023 \%                     & 83                         & 0.526 \%                     & 41    & 0.042 \%& 16   & 0.161 \% & 4     & 0.006 \%\\ \hline
        81-90   & 60                                          & 1.804 \%                                       & 13                                      & 0.025 \%                     & 48                         & 0.304 \%                     & 29    & 0.030 \%& 17   & 0.171 \% & 5     & 0.007 \%\\ \hline
        91-100  & 44                                          & 1.323 \%                                       & 7                                       & 0.013 \%                     & 37                         & 0.234 \%                     & 15    & 0.015 \%& 11   & 0.110 \% & 2     & 0.003 \%\\ \hline
        101-150 & 100                                         & 3.007 \%                                       & 8                                       & 0.015 \%                     & 68                         & 0.431 \%                     & 45    & 0.046 \%& 10   & 0.100 \% & 5     & 0.007 \%\\ \hline
        151-200 & 41                                          & 1.233 \%                                       & 2                                       & 0.003 \%                     & 37                         & 0.234 \%                     & 14    & 0.014 \%& 4    & 0.040 \% & 0     & 0.0 \%\\ \hline
        201-250 & 16                                          & 0.481 \%                                       & 0                                       & 0.0 \%                       & 12                         & 0.076 \%                     & 11    & 0.011 \%& 0    & 0.0 \%   & 0     & 0.0 \%   \\ \hline
        251-300 & 10                                          & 0.300 \%                                       & 0                                       & 0.0 \%                       & 2                          & 0.012 \%                     & 2     & 0.002 \%& 1    & 0.0100 \% & 0     & 0.0 \%  \\ \hline
        300+    & 11                                          & 0.330 \%                                       & 1                                       & 0.0019 \%                    & 4                          & 0.025 \%                     & 7     & 0.007 \%& 0    & 0.0 \%    & 0     & 0.0 \%  \\ \hline
        Avg con & \multicolumn{2}{l|}{42.22}                  & \multicolumn{2}{l|}{2.73}                    & \multicolumn{2}{l|}{19.04}              & \multicolumn{2}{l|}{3.12}  & \multicolumn{2}{l|}{16.72} & \multicolumn{2}{l|}{2.49}                                                  \\ \hline
    \end{tabular}
    \caption{Amount of nodes within a certain node degree for Amazon-Cell-Electronic, Amazon-Cloth-Electronic and Amazon-Cloth-Sport. The Avg connection shows how many connections each user or item have in average}
    \label{tab:node-degrees-cell-sport-electronic}
\end{table*}

\subsubsection{Evaluation Metrics}
We follow the same evaluation metrics as in LightGCN and NGCF \cite{lightgcn,NGCF_2019}.
Each item a user has interacted with in the test set we consider as a positive item, and those they have not interacted with as negative items.
The evaluation metrics are NDCG@50 and Recall@50.
It utilizes the all ranking protocol, where all negative items are candidates.
For ALC and BLC methods are calculated with NDCG@50.
NDCG is a measure for information retrieval where positions are taken into account \cite{Handbook}.
It is calculated from DCG calculated as follows:
\begin{equation}
    NDCG = \frac{DCG}{DCG*},
\end{equation}
where DCG* is the ideal DCG.
DCG is calculated as follows,
\begin{equation}
    DCG = \sum^n_{i=1} \frac{relevance_i}{log_2{(i+1)}}
\end{equation}
where $relevance_i$ is the predicted rating for item $i$, and $n$ is the total amount of items \cite{Handbook,Deep-Learning-on-Graphs-with-GCN}.
Recall is calculated as follows \cite{Handbook}:
\begin{equation}
    Recall = \frac{tp}{tp + fn}
\end{equation}
These evaluation metrics has been chosen because NDCG is well suited to evaluate the priority of recommended items for methods.
Recall is also chosen because it shows how many of the positive items are recommended out of all positive examples in the dataset.


\subsubsection{Baselines}
To see the effectiveness of our methods we compare them to the following baselines:
\begin{itemize}
    \item \textbf{NGCF} \cite{NGCF_2019}: was created for collaborative filtering utilizing a Graph Convolutional Network. Further description can be seen in \autoref{subsubsec:brief-review}.
    \item \textbf{LightGCN} \cite{lightgcn}: was created from NGCF by removing weights, activation function and changing the layer combination to weighted summation. Further description can be seen in \autoref{subsubsec:brief-review}.
    \item \textbf{GCF} \cite{BiTGCF}: is a degenerate method of BiTGCF that does not utilize transfer learning. GCF showed to outperform LightGCN in their experiments.
    \item \textbf{GCN} \cite{GCN}: was created with the purpose of semi-supervised node classification with Graph Convolutional Networks.
    \item \textbf{GC-MC} \cite{GC_MC}: is a graph-based auto-encoder framework for matrix completion that produces latent features for users and items with message passing.
\end{itemize}

\subsubsection{Parameter settings}
BiTGCF and GCF utilized the binary cross entropy loss function as this works well for cross domains, but as we only investigate GCF, we decided to change this to BPR to be able to compare this with LightGCN and NGCF.
For all experiments the embedding size is 64 and mini batch size is 2048 on Yelp2020 and Amazon-Cell-Sport.
Because of the large size of Amazon-Book the mini batch size is set to 8192 to make it faster to process all of the data.
LightGCN is optimized with Adam \cite{Adam} and the learning rate is set to 0.001 and dropout is deactivated.
The embedding parameters are initialized with Xavier method \cite{Xavier,lightgcn}.
The Xavier method initializes the embedding parameters randomly within a certain range to ensure that they are not saturated before the training even starts.
Early stopping is used and the maximum number of epochs is set to 1000.
