\section{Related work}
In this section, we will look at some related work that is relevant to our paper.
We will start by looking at some work related to collaborative filtering and graph-based recommenders.
Finally, we will look at some related work that also extends LightGCN or builds upon LightGCN in some capacity.

\subsection{Collaborative filtering}
Collaborative filtering(CF) is a widely used technique in modern recommender systems \cite{lightgcn}.
There are two main disciplines of CF which are the latent factor approach and the neighborhood approach \cite{SVD}.
Neighborhood approaches compute the relationship between items or alternatively between users.
This will essentially transform users into sets of items which can then be compared.
Latent factor methods represent users and items as parameterized vectors and learn these parameters by reconstructing historical user-item interaction data.
One of the earliest examples of this is Matrix factorization(MF) which maps users and items in a joint latent factor space with the same dimensionality and then models the user-item interactions as inner-products in that space \cite{Matrix-factorization-techniques}.
These representations of users and items are also commonly called embeddings.
MF then uses the dot product of these to predict a user-item interaction.
These methods do not require any domain knowledge because they only represent the users and items by their ID without any additional information.
Newer methods also use neural networks to improve this model while still using the older format of embeddings. 
Examples of this are NCF \cite{NCF} which replaces the inner product with a deep neural network and LRML\cite{LRML} which uses neural networks to improve the learned embeddings.
SVD++ \cite{SVD} combines the neighborhood and the latent factor approach while also taking implicit feedback into account when calculating the predictions.

\subsection{Graph-based recommenders}
Another research area that is relevant to our paper is Graph-based recommenders.
Some of the earlier Graph-Based Recommenders were PageRank, ItemRank and BiRank and ItemRank that utilizes random-walks for recommendation \cite{PageRank,BiRank,ItemRank}.
ItemRank and BiRank uses label propagation where interacted items are labeled and recommended items are based on the similarity between targeted items and interacted items \cite{BiRank,ItemRank}.
In our paper, we use a simplified version of a Neural Graph Collaborative Filtering(NGCF) that is called LightGCN.
NGCF uses GCN's for collaborative filtering \cite{lightgcn, NGCF}.
Simplified Graph Convolutional Networks also tried simplifying GCN's by removing nonlinearities and collapsing weight matrices between consecutive layers, but this method is created for node classification \cite{lightgcn,SGCN}.
Other methods like GCN, GC-MC and GraphSage also utilize graph convolutions to capture the collaborative signals in the user-item interaction graph \cite{GCN,GC_MC,GraphSage}.
For GC-MC it only utilizes one convolutional layer and is focused towards recommendation\cite{GC_MC}, where GCN and GraphSage are used for node classification \cite{GCN,GraphSage}.
Other methods like KGAT and CKAN specialize in using knowledge graphs for collaborative filtering \cite{KGAT,CKAN}.
KGAT propagates embeddings from a node's neighbors while having a specialized attention mechanism that discriminates the importance of different types of neighbors \cite{KGAT}.
CKAN has two different propagation methods, one for collaboration propagation and one for knowledge graph propagation, and then uses a knowledge aware attentive network to differentiate the importance of the different types of relations \cite{CKAN}.


\subsection{Extensions to LightGCN}
There have been multiple extensions to LightGCN and work created inspired by LightGCN \cite{LGC-ACF,IMP-GCN,SGNN,BiTGCF}.
One of them is LightGCN based Aspect-level Collaborative Filtering (LGC-ACF), which utilizes LightGCN and adds side information \cite{LGC-ACF}.
The input to this model is a user-item graph and a user-side information graph. 
These two graphs independently use the propagation method from LightGCN  (as seen on \autoref{eq:lightgcn-propagation}).
For layer combination, LGC-ACF uses weighted summation where all of the layers have equal weights as their aggregation function when combining the user embeddings created from the user-item graph and the user embeddings from the user-side information graph.
This is also done with the item embedding \cite{LGC-ACF}.
Interest-aware Message-Passing GCN (IMP-GCN) is an alternative extension to LightGCN \cite{IMP-GCN}. 
It alleviates the problem of GCNs being over-smoothed by creating clusters within the graphs so that not all nodes will end up being connected.
The result of this method is that it can utilize a higher number of convolutions with good performance compared to LightGCN \cite{IMP-GCN}. 
All of the methods mentioned change or extend the propagation method in LightGCN in some capacity, but none of them try to optimize the layer combination method.
Our methods could therefore be directly used by most of the methods mentioned if they utilize weighted summation.
