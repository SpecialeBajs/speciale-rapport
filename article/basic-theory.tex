\section{Basic Theory}
When dealing with systems with a lot of users and a lot of content it can be beneficial to try and help users find content that they would like.
This is usually done through recommender systems.
There are broadly speaking two approaches to recommender systems which are content-based filtering and collaborative filtering.
Content-based filtering is done by recommending items that are similar to items that the user had like before.
This approach uses features that have to be hand engineered by categorizing the items.
Collaborative filtering on the other hand uses both similarity between users and items to provide recommendations. 
This means that collaborative filtering recommends and item to a user based on a similar user.
The advantage with collaborative filtering is that the embedding's are learned automatically and it does not rely on hand engineered features which makes it more scalable for larger domains with many users and items. 

\subsection{Matrix factorization}


\subsection{Graph Convolutional Network}
