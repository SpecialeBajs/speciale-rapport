\subsection{Optimizing layer combination}
\subsubsection{Removing $\alpha_k$}
% Overvej om denne sektion bør flyttes til appendix med en reference til den.
Before trying to optimize $\alpha_k$ we wanted to investigate what impact it has.
The simplest experiment is to remove it.
If this results in small changes in performance, it can indicate that it would not be worthwhile to pursue the opportunity to optimize $\alpha_k$.
\begin{equation}
    \mathbf{e}_u = \sum_{k=0}^{K} \mathbf{e}_u^{(k)},
    \label{eq:lightgcn-sum}
\end{equation}
The result of the experiment can be seen in \autoref{subsubsec:remove-alpha-k}

\subsubsection{Optimizing $\alpha_k$}
% Definer matematikken
The initial idea of optimizing $\alpha_k$ is to change each layer's effect depending on how many connections each node has.
An example can be seen \autoref{tab:alpha-example}.
The numbers shown on the table are arbitrary and different variables need to be tested to see how well it performs.
The intuition behind this is that each given node needs different kinds of impact.
A node with only 1 connection will not benefit much from the first convolution but will be richer in information in the third convolution.
For a node with 200 connections, the third convolution might actually be harmful, as the node gets impacted too much by too many nodes, where 1 convolution is enough to make it perform optimally.

\begin{table}[]
    \centering
    \begin{tabular}{|l|l|}
        \hline
        Number of connections & Effect of each layer      \\ \hline
        0 - 20                & \begin{tabular}[c]{@{}l@{}}10 \% effect of $e^{(0)}$\\ 15 \% effect of $e^{(1)}$\\ 25 \% effect of $e^{(2)}$\\ 50 \% effect of $e^{(3)}$\end{tabular} \\ \hline
        21-50                 & \begin{tabular}[c]{@{}l@{}}10 \% effect of $e^{(0)}$\\ 10 \% effect of $e^{(1)}$\\ 60 \% effect of $e^{(2)}$\\ 20 \% effect of $e^{(3)}$\end{tabular} \\ \hline
        50+                   & \begin{tabular}[c]{@{}l@{}}10 \% effect of $e^{(0)}$ \\ 50 \% effect of $e^{(1)}$\\ 20 \% effect of $e^{(2)}$ \\ 20 \% effect of $e^{(3)}$\end{tabular} \\ \hline
    \end{tabular}
    \caption{Example of the changed effect to $\alpha_k$. }
    \label{tab:alpha-example}
\end{table}
