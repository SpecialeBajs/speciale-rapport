\subsection{GCF ablation study}
This subsection focus' on answering the following research question: 
\begin{itemize}
    \item \textbf{How do different aggregation functions effect the performance in GCF and LightGCN?}
\end{itemize}
Meng Liu et. al does not present an ablation study for BiTGCF and GCF, which LightGCN showed is important with their regards to NGCF \cite{lightgcn,BiTGCF}.
The GCF aggregation functions is changed by either changing the layer combination to weighted summation, removing inner product, self connections or only utilizing the inner product.
Examples of the changed methods can be seen in the following equations.
\textit{GCF-minus-sc} can be seen on \autoref{eq:GCF-minus-sc} where the self-connection has been removed.
On \autoref{eq:GCF-only-IP} only the inner product of the neighbor has been contained, and is called \textit{GCF-only-IP}.
\autoref{eq:GCF-minus-IP} have removed the inner product and is called \textit{GCF-minus-IP}.
There are also examples where GCF utilizes summation as layer combination as used in LightGCN which can be seen on \autoref{eq:lightgcn-sum}.
The methods that use weighted summation all start with \textit{GCF-sum}.
When LightGCN is tested with concatenation as layer combination it is called \textit{LightGCN-concat}.
\begin{equation}
    \mathbf{e}_{u}^{(k+1)} = \sum^{}_{i \in \mathcal{N}_u}  \frac{1}{\sqrt{|\mathcal{N}_u||\mathcal{N}_i|}}\left( \mathbf{e}_i^{(k)} + \mathbf{e}_i^{(k)} \odot \mathbf{e}_u^{(k)} \right)
    \label{eq:GCF-minus-sc}
\end{equation}
\begin{equation}
    \mathbf{e}_{u}^{(k+1)} = \mathbf{e}_{u}^{(k)} + \sum^{}_{i \in \mathcal{N}_u}  \frac{1}{\sqrt{|\mathcal{N}_u||\mathcal{N}_i|}} \mathbf{e}_i^{(k)} \odot \mathbf{e}_u^{(k)}
    \label{eq:GCF-only-IP}
\end{equation}
\begin{equation}
    \mathbf{e}_{u}^{(k+1)} = \mathbf{e}_{u}^{(k)} + \sum^{}_{i \in \mathcal{N}_u}  \frac{1}{\sqrt{|\mathcal{N}_u||\mathcal{N}_i|}} \mathbf{e}_i^{(k)}
    \label{eq:GCF-minus-IP}
\end{equation}
To reduce redundancy not all methods can be found in equations, but the method names and descriptions of the models are as follows:
\begin{itemize}
    \item \textbf{GCF}: The original GCF method as described in \autoref{subsubsec:GCF-embed-propagation}.
    \item \textbf{GCF-minus-sc}: GCF without self connections.
    \item \textbf{GCF-only-IP}:  GCF where $e_i^{(k)}$ has been removed in \autoref{eq:GCF-embedding}, so that GCF's graph convolutions only considers the inner product of users and items.
    \item \textbf{GCF-only-IP-minus-sc}: Implemented as GCF-only-ip but without self connections.
    \item \textbf{GCF-minus-IP}: GCF where inner product has been removed.
    \item \textbf{LightGCN-concat}: LightGCN with concatenation as layer combination.
    \item \textbf{LightGCN}: Original LightGCN as described in \autoref{subsubsec:LightGCN-embed-propagation}.
    \item \textbf{LightGCN-plus-sc}: LightGCN, but with self connections.
    \item \textbf{GCF-sum-only-IP}: Implemented as GCF-only-IP except that the layer combination method used is weighted summation.
    \item \textbf{GCF-sum}: GCF where the layer combination has been changed to weighted summation instead of concatenation.
    \item \textbf{GCF-sum-minus-sc}: Implemented as GCF-sum but without self connections.
\end{itemize}